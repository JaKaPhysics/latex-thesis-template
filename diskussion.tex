%!TEX root = thesis.tex

\chapter{Diskussion}
\label{chapter:diskussion}

Der folgende Abschnitt ist einem Leitfaden der Universität Bremen \cite{uni-bremen} entnommen:
Häufig fällt es bei der ersten wissenschaftlichen Arbeit schwer, seine Untersuchungsergebnisse zu erörtern bzw. seine Diskussion zu strukturieren. Oft ist es sinnvoll, in der Reihenfolge zu diskutieren, wie die Ergebnisse (Versuche) im Ergebnisteil dargestellt sind. Dies ist aber kein Muss. Ziel sollte jedoch eine Synthese der eigenen Ergebnisse sein. Es lässt sich dabei kaum vermeiden, Teile der Ergebnisse noch einmal aufzugreifen, um daran anzuknüpfen. Jedoch sind reine Ergebnisaufzählungen zu vermeiden. Prinzipiell sollte eine Diskussion folgende Fragen beantworten:

\begin{compactitem}
  \item In welchem Verhältnis stehen meine Ergebnisse zu den bereits bekannten Daten?
  \item Wie lautet die Antwort auf die eingangs formulierten Fragen/ Hypothesen?
  \item Welche Relevanz haben meine Ergebnisse?
\end{compactitem}

Zur Beantwortung dieser Fragen ist es essentiell, sich mit Studien und Ergebnissen anderer Autoren kritisch auseinander zu setzen, diese zu den eigenen Erkenntnissen in Beziehung zu setzen und zu kommentieren. Genau dieser Prozess beschreibt, was wissenschaftliches Schreiben ausmacht. Das Zitieren von relevanter Literatur ist somit unumgänglich. Bereits in der Einleitung haben Sie einen Überblick zu ihrer Thematik gegeben. Jetzt ist es wichtig, einen Vergleich mit ähnlichen Untersuchungen, im Hinblick auf die in der Einleitung aufgeworfenen Fragen und Hypothesen, zu geben. In diesem Kontext wird auch eine kritische Betrachtung der eigenen Methodik etc. erwartet (Fehlerdiskussion). Spekulationen, d.h. Mutmaßungen oder Annahmen über Sachverhalte, sind generell zu vermeiden. Wenn diese aber hilfreich für das Verständnis oder die Gedankenführung sind, müssen diese Meinungen oder Vermutungen als solche erkennbar werden.
Die Diskussion sollte am Ende eine Schlussfolgerung, am besten einen Erkenntnisgewinn, ermöglichen. Diese gibt nicht wieder, was gerade in der Diskussion bewertet wurde, sondern sollte die Diskussion zusammenfassen (was hat diese Untersuchung gebracht und welche Fragen wurden beantwortet) und einen Ausblick auf weiterführende Untersuchungen geben.

